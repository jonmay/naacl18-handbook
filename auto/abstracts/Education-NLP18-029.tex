While immediate feedback on learner language is often discussed in the Second Language Acquisition literature (e.g., Mackey 2006), few systems used in real-life educational settings provide helpful, metalinguistic feedback to learners. In this paper, we present a novel approach leveraging task information to generate the expected range of well-formed and ill-formed variability in learner answers along with the required diagnosis and feedback. We combine this offline generation approach with an online component that matches the actual student answers against the pre-computed hypotheses. The results obtained for a set of 33 thousand answers of 7th grade German high school students learning English show that the approach successfully covers frequent answer patterns. At the same time, paraphrases and content errors require a more flexible alignment approach, for which we are planning to complement the method with the CoMiC approach successfully used for the analysis of reading comprehension answers (Meurers et al., 2011).
