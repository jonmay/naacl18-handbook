We present the first work on predicting reading mistakes in children with reading difficulties based on eye-tracking data from real-world reading teaching. Our approach employs several linguistic and gaze-based features to inform an ensemble of different classifiers, including multi-task learning models that let us transfer knowledge about individual readers to attain better predictions. Notably, the data we use in this work stems from noisy readings in the wild, outside of controlled lab conditions. Our experiments show that despite the noise and despite the small fraction of misreadings, gaze data improves the performance more than any other feature group and our models achieve good performance. We further show that gaze patterns for misread words do not fully generalize across readers, but that we can transfer some knowledge between readers using multitask learning at least in some cases. Applications of our models include partial automation of reading assessment as well as personalized text simplification.
