Input material at the appropriate level is crucial for language acquisition. Automating the search for such material can systematically and efficiently support teachers in their pedagogical practice. This is the goal of the computational linguistic task of automatic input enrichment (Chinkina \& Meurers, 2016): It analyzes and re-ranks a collection of texts in order to prioritize those containing target linguistic forms. In the online study described in the paper, we collected 240 responses from English teachers in order to investigate whether they preferred automatic input enrichment over web search when selecting reading material for class. Participants demonstrated a general preference for the material provided by an automatic input enrichment system. It was also rated significantly higher than the texts retrieved by a standard web search engine with regard to the representation of linguistic forms and equivalent with regard to the relevance of the content to the topic. We discuss the implications of the results for language teaching and consider the potential strands of future research.
