When interpreting questions in a virtual patient dialogue system one must inevitably tackle the challenge of a long tail of relatively infrequently asked questions.  To make progress on this challenge, we investigate the use of paraphrasing for data augmentation and neural memory-based classification, finding that the two methods work best in combination. In particular, we find that the neural memory-based approach not only outperforms a straight CNN classifier on low frequency questions, but also takes better advantage of the augmented data created by paraphrasing, together yielding a nearly 10\% absolute improvement in accuracy on the least frequently asked questions.
