Conversational AI systems, such as Amazon's Alexa, are rapidly developing from purely transactional systems to social chatbots, which can respond to a wide variety of user requests. In this article, we establish how current state-of-the-art conversational systems react to inappropriate requests, such as bullying and sexual harassment on the part of the user, by collecting and analysing the novel \#MeTooAlexa corpus. Our results show that commercial systems mainly avoid answering, while rule-based chatbots show a variety of behaviours and often deflect. Data-driven systems, on the other hand, are often non-coherent, but also run the risk of being interpreted as flirtatious and sometimes react with counter-aggression. This includes our own system, trained on ``clean'' data, which suggests that inappropriate system behaviour is not caused by data bias.
