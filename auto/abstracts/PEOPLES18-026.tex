Over the years, natural language processing has increasingly focused on tasks that can be solved by statistical models, but ignored the social aspects of language. These limitations are in large part due to historically available data and the limitations of the models, but have narrowed our focus and biased the tools demographically. However, with the increased availability of data sets including socio-demographic information and more expressive (neural) models, we have the opportunity to address both issues. I argue that this combination can broaden the focus of NLP to solve a whole new range of tasks, enable us to generate novel linguistic insights, and provide fairer tools for everyone.
