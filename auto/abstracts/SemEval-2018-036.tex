In this paper we describe our participation in the SemEval-2018 task 3 Shared Task on Irony Detection. We have approached the task with our low dimensionality representation method (LDR), which exploits low dimensional features extracted from text on the basis of the occurrence probability of the words depending on each class. Our intuition is that words in ironic texts have different probability of occurrence than in non-ironic ones. Our approach obtained acceptable results in both subtasks A and B. We have performed an error analysis that shows the difference on correct and incorrect classified tweets.
