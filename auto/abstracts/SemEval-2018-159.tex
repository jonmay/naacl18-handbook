Emoji is one of the ''fastest growing language '' in pop-culture, especially in social media and it is very unlikely for its usage to decrease. These are generally used to bring an extra level of meaning to the texts, posted on social media platforms. Providing such an added info, gives more insights to the plain text, arising to hidden interpretation within the text. This paper explains our analysis on Task 2, '' Multilingual Emoji Prediction'' sharedtask conducted by Semeval-2018. In the task, a predicted emoji based on a piece of Twitter text are labelled under 20 different classes (most commonly used emojis) where these classes are learnt and further predicted are made for unseen Twitter text. In this work, we have experimented and analysed emojis predicted based on Twitter text, as a classification problem where the entailing emoji is considered as a label for every individual text data. We have implemented this using distributed representation of text through fastText. Also, we have made an effort to demonstrate how fastText framework can be useful in case of emoji prediction. This task is divide into two subtask, they are based on dataset presented in two different languages English and Spanish.
