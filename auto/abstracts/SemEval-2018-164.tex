This paper discusses the experiments performed for predicting the emotion intensity in tweets using a generalized supervised learning approach. We extract 3 kind of features from each of the tweets - one denoting the sentiment and emotion metrics obtained from different sentiment lexicons, one denoting the semantic representation of the word using dense representations like Glove, Word2vec and finally the syntactic information through POS N-grams, Word clusters, etc. We provide a comparative analysis of the significance of each of these features individually and in combination tested over standard regressors avaliable in scikit-learn. We apply an ensemble of these models to choose the best combination over cross validation.
