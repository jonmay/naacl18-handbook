Developing computational models of spatial prepositions (such as on, in, above, etc.) is crucial for such tasks as human-machine collaboration, story understanding, and 3D model generation from descriptions. However, these prepositions are notoriously vague and ambiguous, with meanings depending on the types, shapes and sizes of entities in the argument positions, the physical and task context, and other factors. As a result truth value judgments for prepositional relations are often uncertain and variable. In this paper we treat the modeling task as calling for assignment of probabilities to such relations as a function of multiple factors, where such probabilities can be viewed as estimates of whether humans would judge the relations to hold in given circumstances. We implemented our models in a 3D blocks world and a room world in a computer graphics setting, and found that true/false judgments based on these models do not differ much more from human judgments that the latter differ from one another. However, what really matters pragmatically is not the accuracy of truth value judgments but whether, for instance, the computer models suffice for identifying objects described in terms of prepositional relations, (e.g., ``the box to the left of the table'', where there are multiple boxes). For such tasks, our models achieved accuracies above 90\% for most relations.
