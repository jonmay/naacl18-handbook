Natural language text exhibits hierarchical structure in a variety of respects. Ideally, we could incorporate our prior knowledge of this hierarchical structure into unsupervised learning algorithms that work on text data. Recent work by \citet{nickel2017poincar} proposed using hyperbolic instead of Euclidean embedding spaces to represent hierarchical data and demonstrated encouraging results when embedding graphs. In this work, we extend their method with a re-parameterization technique that allows us to learn hyperbolic embeddings of arbitrarily parameterized objects. We apply this framework to learn word and sentence embeddings in hyperbolic space in an unsupervised manner from text corpora. The resulting embeddings seem to encode certain intuitive notions of hierarchy, such as word-context frequency and phrase constituency. However, the implicit continuous hierarchy in the learned hyperbolic space makes interrogating the model's learned hierarchies more difficult than for models that learn explicit edges between items. The learned hyperbolic embeddings show improvements over Euclidean embeddings in some -- but not all -- downstream tasks, suggesting that hierarchical organization is more useful for some tasks than others.
