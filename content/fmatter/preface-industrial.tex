\vspace{2em}

\section{Message from the Industry Track Co-Chairs}\vspace{2em}
\setheaders%
    {Message from the Industry Track Co-Chairs}%
    {Message from the Industry Track Co-Chairs}
\thispagestyle{emptyheader}

\setlength{\parskip}{1ex}

\textbf{OLD PLACEHOLDER TEXT FOLLOWS}

It is our pleasure to welcome you to the inaugural Industry Track in the *ACL family of conferences at NAACL 2018. 

The idea of organizing an industry track stemmed from the challenging issues encountered while attempting to apply state-of-the-art techniques to real-world language problems. As those who have attempted these problems know, practical applications are rarely as well defined as in laboratory settings and the data never as clean. In addition, there may be practical constraints such as computational requirements, processing speed, memory footprint, latency requirements, ease of updating a deployed solution that need to be balanced judiciously, and capability to be embedded as part of a larger system. The NAACL 2018 Industry Track was born out of the desire to provide a forum for researchers, engineers and application developers who are experienced in these issues to exchange ideas, share results and discuss use-cases of successful development and deployments of language technologies in real-world settings. 

Although we thought that the time is ripe in the NLP field for such a forum, and hoped that the community will embrace the opportunity to share their experience with others, it was nonetheless a guessing game as to the amount of interest the track would actually generate. As submissions drew to a close in late February, we were happy to report that we received 91 submissions, far exceeding our expectations (which led to last-minute scrambling to recruit more reviewers, but we're not complaining!). Six of the papers were desk rejects due to non-conformance with submission requirements, and the remaining 85 papers were reviewed by 65 reviewers. We accepted 28 papers -- an acceptance rate of 32.9\% (one paper was subsequently withdrawn after acceptance) of which 19 papers will be presented in oral sessions that run as a parallel track during the main conference, and 8 papers will be presented during poster sessions. Of course, none of this would have been possible without the participation of authors and reviewers, and we would like to convey our heartfelt "thank you" to all the authors for submitting papers and the reviewers for their efforts in the paper selection process.

We analyzed our submissions along a couple of dimensions and would like to share some interesting statistics. First we looked at the submissions with respect to the distribution of author affiliations. As one would expect, the industry track focuses on problems that manifest themselves more readily in industry than in academia. Indeed, of the 85 papers reviewed, 55 papers are authored by researchers/engineers in industry laboratories. The particularly encouraging statistic, however, is that 25 papers are the results of collaboration between those in industry and academia. it would be interesting to track these statistics in future years to see if the collaboration increases as the field continues to mature. The second dimension we analyzed is the geographic distribution of authors by contact author. This being a NAACL conference, it is no surprise that 62\% of the papers came from North America. We are pleased with the participation of authors from other regions, including 22\% from Europe, 14\% from Asia, and 1\% from Africa. 

In addition to paper presentations, we will have two plenary keynote speeches. For the keynote speeches, we aimed to feature researchers who also have first hand experience applying research results to practical applications. To that end, we are honored to have two illustrious members of NLP community --  Daniel Marcu, who co-founded Language Weaver more than 15 years ago and is now the director of MT/NLP at Amazon, and Mari Ostendorf, professor at the University of Washington, who led a team of students to build a social bot that won the 2017 Alexa Prize competition. We are confident that their experiences would be of immense interest to the larger NLP community.

Another highlight of the industry track includes two panel discussions on topics of increasing importance in the community. The first panel, "Careers in Industry", moderated by Philip Resnik, professor at University of Maryland, is primarily geared toward students and recent graduates who are exploring careers in industry versus academia. The panel will feature experienced professionals who have worked in both environments to share their experience and offer advice, based on questions gathered from the *ACL community earlier this year . The second panel, "Ethics in NLP", will be moderated by Dirk Hovy, professor at Università Bocconi, and will focus on raising awareness of the emerging issues of biases present in NLP/AI solutions, the social implications of such biases, and what we, as NLP practitioners, can do to reduce them. 

With the overwhelming response to the call for papers, the language community has unambiguously endorsed the relevance of the Industry track in the milieu of annual conferences.  As organizers, we have attempted to amplify this endorsement by bringing to the participants a invigorating technical program. We hope through your engaging discussions and active participation during the sessions, you will unanimously support and nurture the concept of an Industry track in NLP conferences over the years to come.



\vskip 0.5in
\noindent Srinivas Bangalore (Interactions Labs) \\
Jennifer Chu-Carroll (Elemental Cognition) \\
Yunyao Li (IBM Research - Almaden) \\
NAACL 2018 Industry Track Co-Chairs

\index{Bangalore, Srinivas}
\index{Chu-Carroll, Jennifer}
\index{Li, Yunyao}


